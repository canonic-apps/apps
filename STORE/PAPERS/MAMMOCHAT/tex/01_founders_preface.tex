% ============================================================
% 01_founders_preface
% ============================================================

\begin{strip}
\centering
\begin{minipage}[t]{0.48\textwidth}
	\begin{tcolorbox}[redvignette={\textbf{Maria - 47-year-old woman}}, equal height group=vignettes]
		\RaggedRight\small
		\textit{Maria is a 47-year-old Catholic schoolteacher from Colombia who presented after a screening mammogram revealed a unilateral asymmetry.}
Biopsy confirmed \textbf{stage I luminal A breast cancer}.
She underwent \textbf{breast-conserving surgery} and adjuvant endocrine therapy without complication.
After her divorce, she relocated to Florida to be closer to family but reported feeling \textbf{isolated, anxious, and fearful}.
Her electronic health portals were in English only, and fragmented communication added to her distress.
Social screening was consistent with \textbf{moderate anxiety} (GAD-7 = 13) and \textbf{mild depression} (PHQ-9 = 10).
	\end{tcolorbox}
\end{minipage}\hfill
\begin{minipage}[t]{0.48\textwidth}
	\begin{tcolorbox}[redvignette={\textbf{Zaida - 52-year-old woman}}, equal height group=vignettes]
		\RaggedRight\small
		\textit{Zaida is a 52-year-old software engineer of Pakistani heritage, an observant Muslim who lives alone and works remotely.}
She presented after noticing axillary fullness, and biopsy revealed \textbf{node-positive, HER2-positive invasive ductal carcinoma}.
She began \textbf{neoadjuvant HER2-directed therapy} with plans for surgery and adjuvant radiation.
Although her treatment was technologically advanced-with wearables, remote vitals, and EHR-linked data-Zaida described feeling \textbf{ watched but not understood. }
Frequent alerts disrupted her work and prayer routines, amplifying a sense of \textbf{fear and isolation}.
	\end{tcolorbox}
\end{minipage}

\captionof{figure}{Dual clinical vignettes illustrating the emotional and technological challenges faced by breast cancer survivors. Maria (left) and Zaida (right) represent contrasting experiences of isolation and anxiety-one from language and cultural barriers, the other from technology overload. Both highlight the need for empathy-native digital health solutions.}
\end{strip}

\section{Founder's Preface: From Awareness to Agency}

\textbf{Maria and Zaida} both experienced fear, anxiety, and isolation during breast-cancer care.
Both cases reflect how breast-cancer survivors can experience profound \textbf{emotional distress} despite excellent clinical care.
Maria's challenge stemmed from \textbf{silence and disconnection} through language and cultural barriers, whereas Zaida experienced mainly \textbf{noise and intrusion} through technology overload.
For both women, the dominant theme was \textbf{fear of being unseen} that was only amplified by an inherent lack of empathy in navigating a complex health system.

\textbf{The Problem.}
When I first understood women like \textbf{Maria} and \textbf{Zaida}, I realized they were two victims of the same broken health system.
Maria's silence came from language-a maze of English-only portals and clinical jargon that left her unseen.
Zaida's noise came from automation-wearables, dashboards, and constant alerts that left her unheard.
Both experienced fear and isolation within systems built to measure but not to understand.
MammoChat was founded to change that: to make compassion as quantifiable as precision, and connection as powerful as any biomarker.

\textbf{The Vision.}
MammoChat fuses \textbf{clinical medicine, artificial intelligence, and empathy by design.}
It doesn't replace clinicians-it restores context.
Every feature follows one organizing principle: \textit{listen first, explain clearly, act last}.
That discipline drives the \textbf{OPTS-EGO Ledger}, a foundation that transforms compliance into capability by linking trust, transparency, and consent into a single chain of care.
Behind MammoChat's conversational interface lies the \textbf{mCODE} framework, which organizes oncology data across institutions, and the \textbf{OPTS-EGO Ledger}, which secures consent and provenance for every shared record.\autocite{fhir_provenance,prov_o_w3c,ga4gh_policy,grossman2018cellsystems}
Together, they turn fragmented experiences into a continuous, patient-authored narrative of care\autocite{robinson2020bmjqs}.

\textbf{Empathy as Architecture.}
In most AI systems, empathy is an interface; in MammoChat, it s infrastructure.
Our conversational model validates emotion before delivering information-a design choice proven to improve retention, trust, and follow-through.
For Maria, a bilingual chat interface and faith-aligned peer network restored agency and belonging.
For Zaida, adaptive notifications and privacy-respecting scheduling reduced digital fatigue while honoring her faith practices.
Empathy is not ornamental.
It is structural-encoded into how data is interpreted, visualized, and shared.

\textbf{From Compliance to Capability.}
For decades, privacy and provenance were seen as burdens.
We see them as the engines of participation.
The \textbf{OPTS-EGO Ledger} records every mammogram, biopsy, and lab note as a verified, patient-owned contribution to science.
OPTS (Open Provenance Token Standard) ensures traceable data lineage;
EGO (Ethical Governance Operators) secures integrity and consent validation.
Together they transform records into relationships-creating the technical trust layer precision medicine has always needed.

\textbf{From Awareness to Ownership.}
Breast Cancer Awareness Month taught the world to care; now it must teach the world to act.
MammoChat converts awareness into ownership by giving women agency over how their data, stories, and insights shape research.
Through the \textbf{mCODE} standard, we unify oncology data across systems;\autocite{ga4gh_vr}
through empathy-native design, we unify patient experience across cultures
The result is a platform where survivors become collaborators, and participation becomes the new prevention.
Maria and Zaida's stories represent two sides of the same truth: survivorship requires more than data and treatment-it demands understanding.

\textbf{Policy into Practice.}
Every layer of MammoChat aligns with the public infrastructure of trust built over two decades-from HIPAA and HITECH to the 21st Century Cures Act and NIH's Data Sharing Policy.
What was once a paper right is now a living mechanism for transparency.
By merging provenance with empathy, we bridge compliance and care, creating an \textbf{open, ethical, and auditable ecosystem} that turns regulation into resilience.

\textbf{The Call Ahead.}
These cases anchor MammoChat's \textbf{\$12 million Series A} strategy to reach \textbf{20\,000 women and their providers} across partner health systems in Florida and California.
Funding will scale empathy-native modules, integrate mCODE and OPTS-EGO into EHRs, and expand peer navigation across faith and language groups.
But the true horizon is larger than a market-it is a movement.
We are building the infrastructure of empathy: a shared digital commons where data is honored, stories are preserved, and trust compounds with every interaction\autocite{grossman2018cellsystems}.
Every hospital, advocacy group, and survivor community has a role in shaping this ecosystem, where empathy and evidence are co-equal forms of care.
By uniting clinicians, technologists, and patients under a common data language of trust, MammoChat invites partners to \textbf{join us on this journey}-to co-create a future where care listens before it learns, and every woman is not just treated, but truly \textbf{understood, seen, and remembered.}

\textbf{Dexter Hadley, MD/PhD}\\
Founder \& CEO, MammoChat\texttrademark\\
Lake Nona Medical City, Florida\\
October 31, 2025