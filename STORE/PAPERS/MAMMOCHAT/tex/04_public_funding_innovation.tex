% ============================================================
% 04_public_funding_innovation.tex
% ============================================================

\section{Public Funding Provenance Innovation}
\label{sec:public-funding}
Public investment in science is society s boldest act of faith.
Every grant carries an expectation that knowledge created with public dollars should, in turn, serve the public good.
Yet for decades, tracing where that knowledge goes - and who benefits - has remained opaque.
MammoChat's \textbf{OPTS-EGO Ledger} transforms this expectation into a measurable framework: every publicly funded deliverable becomes a verified, reusable, and attributable digital artifact.

\textbf{The Casey DeSantis Florida Cancer Innovation Grant.}
In 2025, the State of Florida awarded a \$2 M Cancer Innovation Grant under the leadership of First Lady Casey DeSantis.
The grant s objective: to accelerate translation of digital health innovation into community cancer prevention and survivorship.
Its mandate was explicit - all resulting code, datasets, and intellectual outputs must be made open source and accessible to the public.

For many institutions, this requirement is administrative; for MammoChat, it became a design principle.
Each deliverable is now minted as an \textbf{OPTS-Grant Token} - a cryptographic certificate that records authorship, funding source, and time of creation.

\textbf{The Provenance Chain of a Grant Deliverable.}

\begin{figure}
\centering
\resizebox{0.95\textwidth}{!}{%
\begin{tikzpicture}[node distance=1.4cm, >=Latex]
\tikzstyle{n}=[rectangle, draw=slate, fill=mammoPink!10,
rounded corners, minimum width=8em, minimum height=1.8em, text centered]

\node[n] (fund) {Public Funding};
\node[n, right=3.2cm of fund] (research) {Research Output};
\node[n, right=3.2cm of research] (token) {OPTS--Grant Token};
\node[n, right=3.2cm of token] (reuse) {Innovation Reuse};
\node[n, right=3.2cm of reuse] (benefit) {Patient Benefit};

\draw[->, thick] (fund) -- node[above]{\small Award + Conditions} (research);
\draw[->, thick] (research) -- node[above]{\small Hash + Sign} (token);
\draw[->, thick] (token) -- node[above]{\small Register + Cite} (reuse);
\draw[->, thick] (reuse) -- node[above]{\small Deploy + Measure} (benefit);
\end{tikzpicture}}
\caption{Lifecycle of a publicly funded deliverable within the OPTS--EGO Ledger. Each stage builds accountability: from public investment to measurable patient benefit.}
\end{figure}

Every dataset, model, or line of code is hashed using SHA-3, signed by its principal investigator, and recorded as an immutable entry on the ledger The result is a transparent and permanent record linking public investment to scientific and clinical benefit.

\textbf{Transparency by Design.}
Traditional reporting models rely on periodic progress summaries and post-hoc compliance checks
The OPTS-Grant Registry replaces this with continuous verification.
Each ledger entry contains:
\begin{itemize}
 \item the funding source (e.g., State of Florida Cancer Innovation Program);
 \item the principal investigator s digital signature;
 \item the SHA-3 hash of the deliverable (e.g., code repository, dataset);
 \item metadata describing the scope, license, and version;
 \item timestamps for creation and publication.
\end{itemize}

Anyone - auditor, policymaker, or patient - can confirm that a deliverable funded by public money has been created, shared, and preserved.
This moves transparency from legal compliance to cryptographic proof.

\textbf{MammoChat's Demonstration Deliverables.}
\begin{itemize}
 \item \textbf{D1 - mCODE Translator:} converts unstructured clinical text and imaging metadata into mCODE-compliant JSON.
 \item \textbf{D2 - Community Empathy Model:} a multilingual conversational agent trained to deliver emotional support aligned with medical literacy standards.
 \item \textbf{D3 - OPTS-EGO Prototype Ledger:} open-source blockchain module for consent recording and provenance tracking.
\end{itemize}

Each is registered as an OPTS-Grant Token with open access for reuse by clinicians, developers, and patient advocates.

\textbf{Ethical Governance and Attribution.}
In traditional academia, authorship disputes and data hoarding often limit collaboration
By contrast, OPTS-Grant Tokens formalize credit at the level of contribution.
If a future startup or research team reuses MammoChat's open-source components, attribution and citation are built into the ledger itself.
The result is a self-auditing ecosystem that rewards openness rather than secrecy.\autocite{goldfarb2019book}

\textbf{Economic Flow.}
The model also closes the loop between public funding and patient benefit.

\medskip
\noindent
\begin{minipage}{\linewidth}
\raggedright
\textit{Public Investment} $\rightarrow$
\textit{Open Deliverables} $\rightarrow$
\textit{Innovation Reuse} $\rightarrow$
\textit{Improved Outcomes} $\rightarrow$
\textit{Public Trust.}
\end{minipage}

\medskip
Every reuse generates measurable social and economic return, documented through ledger analytics rather than grant reports.

\textbf{Why It Matters.}
When patients like Maria and Zaida access a tool built with state funds, they become part of a transparent continuum - one where their participation feeds directly back into the system that supported them.
The \$2 M Casey DeSantis Grant thus serves as both catalyst and proof-of-concept: a living example of how provenance transforms public spending into perpetual innovation.
Therefore, any similar public-facing reporting, evidence, or evaluation frameworks may adopt our open-source approach to achieve transparent operational efficiency.

\begin{quote}\centering
\textit{ Public trust is the ultimate currency of healthcare.
Provenance is how we earn it back. }
\end{quote}